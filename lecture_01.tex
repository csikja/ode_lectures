\documentclass[10pt]{beamer}
%\usepackage[utf8]{inputenc}

\usetheme[progressbar=frametitle]{metropolis}
\usepackage{appendixnumberbeamer}
\usepackage{booktabs}
\usepackage[scale=2]{ccicons}

\usepackage{amsmath, amssymb, amsfonts}

\usepackage{pgfplots}
\usepgfplotslibrary{dateplot}

\usepackage{xspace}
\newcommand{\themename}{\textbf{\textsc{metropolis}}\xspace}

\usepackage[sfdefault, lining]{FiraSans}
\renewcommand*\oldstylenums[1]{{\firaoldstyle #1}}
\usepackage{newtxsf}
\usepackage{microtype}

\usepackage{commath}

\newcommand{\R}[0]{\mathbb{R}}
\newcommand{\bp}[0]{\mathbf{p}}
\newcommand{\be}[0]{\mathbf{e}}
\newcommand{\bv}[0]{\mathbf{v}}
\newcommand{\bF}[0]{\mathbf{F}}
\DeclareMathOperator{\sign}{sign}
\newcommand{\dlt}{\Delta t}

\title{Differenciálegyenletek}
\subtitle{Modellezés differenciálegyenletekkel}
% \date{\today}
\date{}
\author{Csikja Rudolf}
\institute{Budapesti Műszaki és Gazdaságtudományi Egyetem\\Matemaikai Intézet, Analízis Tanszék}
% \titlegraphic{\hfill\includegraphics[height=1.5cm]{logo.pdf}}

\begin{document}

\maketitle

%\begin{frame}{Table of contents}
%  \setbeamertemplate{section in toc}[sections numbered]
%  \tableofcontents%[hideallsubsections]
%\end{frame}

\section{Közönséges differenciálegyenletek}
\begin{frame}[t]{Matematikai modellezés}
\[x(t+\dlt) = x(t) + f(t, x(t))\dlt + \varepsilon(\dlt) \dlt\]
 \[\lim_{\dlt \to 0} \varepsilon(\dlt) = 0\]
 \[\lim_{\dlt \to 0} \frac{x(t+\dlt) - x(t)}{\dlt} = f(t, x(t)) + \lim_{\dlt \to 0} \varepsilon(\dlt)\]
  \[x'(t) = f(t, x(t))\]
\end{frame}

\section{Populációdinamika}
\begin{frame}[t]{Exponenciális növekedés}
\[x'(t) = k x(t)\]
\end{frame}

\begin{frame}[t]{Korlátozott növekedés}
Annak érdekében, hogy figyelembe vegyük a véges erőforrást
az exponenciális növekedés modeljében a növekedési rátát módosítjuk.
A növekedési rátát függővé tesszük a populáció nagyságától:
\[x'(t) = K(x(t)) x(t).\]
\begin{itemize}
\item Továbbra is elvárjuk, hogy kicsi populációra a növekedési ráta ugyanaz legyen,
mint az expononenciális növekedés modeljében: $K(0) = k.$
\item Ugyanakkor azt is elvárjuk, hogy egy bizonyos populáció elérésénél a növekedési
ráta nulla legyen: $K(L) = 0.$
\end{itemize}
A lehető legegyszerűbb függvény, ami megfelel a fenti feltételeknek a
\[K(p) = k - \frac{k}{L}p\]
egyenes.
\end{frame}

\begin{frame}[t]{Korlátozott növekedés}
 \[x'(t) = k x(t)\left(1 - \frac{x(t)}{L}\right)\]
\end{frame}



\section{Klasszikus mechanika}
\begin{frame}[t]{Newton-féle mozgásegyenlet}
Newton II. törvénye szerint egy test $\bp=m\bv$ lendületének (időbeli) megváltozása
arányos a testre ható erővel $\bp'(t) = \bF(t).$
A sebességre vonatkozó differenciálegyenlet így
\[\bv'(t) = \frac{1}{m}\bF(t)\]
\end{frame}

\begin{frame}[t]{Centrális erőtér}
\[F = \]
\[\bF = \begin{bmatrix}
         F_x\\ F_y
        \end{bmatrix}
        \sim 
        F = F_x + i F_y
\]
\[z = x + iy = r e^{i \theta}\]
\[z' = r' e^{i \theta} + i r \theta' e^{i \theta}\]
\[z'' = r'' e^{i \theta} + 2 i r' \theta' e^{i \theta} + i r \theta'' e^{i \theta} - r \theta'^2 e^{i\theta}\]
\[\frac{\theta''}{\theta'} = -2\frac{r'}{r}\quad  \to \quad \theta' = \frac{1}{r^2}\]

\end{frame}


\begin{frame}[t]{Gravitáció}
\[
\bF_{12} = G \frac{m_1 m_2}{r^2}\be_r
\]

\[F_g = GM \frac{m}{r^2}\]

\[E_p(x) = \int_{R}^x \frac{mMg}{r^2}\, \dif r\]

Szökési sebesség:
\[\frac{1}{2}m v_{\infty}^2 = E_p(\infty) = GM\frac{m}{R}\]
\[v_\infty = \sqrt{\frac{2 GM}{R}}\approx 11.186\, \mathsf{km/s}\]

\end{frame}

\begin{frame}[t]{Szabadesés}
\[
\begin{split}
v'(t)   &= \frac{1}{m}F_g = g
\end{split}
\]
hence
\[v(t) = v(0) + g t\]
Másfél perces esés végére a sebesség meghaladná a $3000\,$km/h-t.
\end{frame}

\begin{frame}[t]{Szabadesés légellenállással}
\begin{columns}
\column{0.5\textwidth}
    \[
    v'(t) = -\frac{1}{m}F_g + \frac{1}{m}F_D(v(t))
    \]
    a légellenállásból származó erő
    \[
    F_D(v) = -\frac{1}{2}\rho C_D A v^2 \sign(v)
    \]
    \begin{center}
    \includegraphics[width=0.8\textwidth]{drag_force.pdf}
    \end{center}
    \[v'(t) = -g + \frac{\rho C_D A}{2m} v^2(t), \quad v(t)\le 0.\]
\column{0.5\textwidth}
    \[v'(t) = -g\left(1-\left(\frac{v(t)}{V_T}\right)^2\right)\]
    A végsebesség feltétele $v'(t) = 0,$ amiből $V_{T} = \pm\sqrt{\frac{2 g m}{\rho C_D A}}.$
    \begin{center}
        \includegraphics[scale=0.05]{skydive.png}
    \end{center}
    Az eredeti egyenlet felírható így:
Megoldás:
\[v(t) = V_T \left(\frac{2}{ 1 + \frac{V_T - v_0}{V_T + v_0}e^{\frac{2g}{V_T}t }} - 1\right)\]
\end{columns}
\end{frame}



\end{document}
