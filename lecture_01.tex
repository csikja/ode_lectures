\documentclass[10pt]{beamer}
%\usepackage[utf8]{inputenc}

\usetheme[progressbar=frametitle]{metropolis}
\usepackage{appendixnumberbeamer}

\usepackage{booktabs}
\usepackage[scale=2]{ccicons}

\usepackage{pgfplots}
\usepgfplotslibrary{dateplot}

\usepackage{xspace}
\newcommand{\themename}{\textbf{\textsc{metropolis}}\xspace}

\usepackage[sfdefault, lining]{FiraSans}
\renewcommand*\oldstylenums[1]{{\firaoldstyle #1}}
\usepackage{newtxsf}
\usepackage{microtype}

\newcommand{\R}[0]{\mathbb{R}}
\newcommand{\bp}[0]{\mathbf{p}}
\newcommand{\bv}[0]{\mathbf{v}}
\newcommand{\bF}[0]{\mathbf{F}}

\title{Differenciálegyenletek}
\subtitle{Bevezetés az elméletbe és alkalmazásokba}
% \date{\today}
\date{}
\author{Csikja Rudolf}
\institute{Budapesti Műszaki és Gazdaságtudományi Egyetem\\Matemaikai Intézet, Analízis Tanszék}
% \titlegraphic{\hfill\includegraphics[height=1.5cm]{logo.pdf}}

\begin{document}

\maketitle

%\begin{frame}{Table of contents}
%  \setbeamertemplate{section in toc}[sections numbered]
%  \tableofcontents%[hideallsubsections]
%\end{frame}


\section{Modellezés differenciálegyenletekkel}
\subsection{Populáció dinamika}
\begin{frame}[t]{Exponenciális mövekedés}
\[x(t+\Delta t) = x(t) +  \Delta t\cdot k x(t) + \varepsilon(\Delta t), \qquad \lim_{\Delta t\to 0} \varepsilon(\Delta t) = 0\]
\end{frame}


\subsection{}
\begin{frame}[t]{Newton féle mozgásegyenletek}
Newton II. törvénye szerint egy test lendületének (időbeli) megváltozása
arányos a testre ható erővel $\dot \bp = \bF.$
Feltéve, hogy a tömeg állandó $\bp=m\bv$

\[v(t) = v(0) + \frac{1}{m}\int_0^t F(\tau)\, d\tau\]
\end{frame}


\end{document}
