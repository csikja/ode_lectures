\documentclass[10pt]{beamer}
%\usepackage[utf8]{inputenc}

\usetheme[progressbar=frametitle, block=fill]{metropolis}
\usepackage{appendixnumberbeamer}

\usepackage{booktabs}
\usepackage[scale=2]{ccicons}

\usepackage{enumitem, multicol}
\usepackage{amsmath, amssymb, amsfonts}

\usepackage{pgfplots}
\usepgfplotslibrary{dateplot}


\usepackage{xspace}
\newcommand{\themename}{\textbf{\textsc{metropolis}}\xspace}
% \usecolortheme{seahorse}

\usepackage[sfdefault, lining]{FiraSans}
\renewcommand*\oldstylenums[1]{{\firaoldstyle #1}}
\usepackage{newtxsf}
\usepackage{microtype}

\renewcommand{\emph}[1]{{\bf #1}}

\newcommand{\R}[0]{\mathbb{R}}
\newcommand{\C}[0]{\mathbb{C}}
\newcommand{\bp}[0]{\mathbf{p}}
\newcommand{\bv}[0]{\mathbf{v}}
\newcommand{\bF}[0]{\mathbf{F}}
\DeclareMathOperator{\sign}{sign}

\title{Differenciálegyenletek}
\subtitle{Előismeretek}
% \date{\today}
\date{}
\author{Csikja Rudolf}
\institute{Budapesti Műszaki és Gazdaságtudományi Egyetem\\Matemaikai Intézet, Analízis Tanszék}
% \titlegraphic{\hfill\includegraphics[height=1.5cm]{logo.pdf}}

\begin{document}

\maketitle

%\begin{frame}{Table of contents}
%  \setbeamertemplate{section in toc}[sections numbered]
%  \tableofcontents%[hideallsubsections]
%\end{frame}


\section{Analízis}
\begin{frame}[t]{Függvények}
A függvény egy \emph{hozzárendelési szabály}, például az
$f\colon\R\to\R$ skalár függvény vagy a $g\colon\R\to\R^n$ vektor függvény.

Az $f$ függvény értéke az $x\in\R$ helyen egy \emph{valós szám}: $f(x)\in\R.$
A $g$ függvény értéke a $t\in\R$ helyen $g(t)\in\R^n$, egy $n$-dimenziós \emph{vektor}.

A hozzárendelési szabályt gyakran formulával adjuk meg, például
\[f(x) = \sin(x), \quad g(t) = \begin{bmatrix}3t^2 - t + 1\\ 1 + t\ln(t)\end{bmatrix}.\]
Egy függvénynek van \emph{értelmezési tartománya} (és értékkészlete is).
Sőt, az értelmezési tartomány a függvény definíciójának része!
Az
\[x\mapsto x^2 \quad (x\in\R) \qquad \text{és} \qquad x \mapsto x^2\quad (x\in\R^+)\]
függvények nem azonosak.
\end{frame}

\begin{frame}[t]{Függvények: modellezés}
A függvények alkalmasak fizikai változók modellezésére.
Például egy kemence hőmérsékletét az időben jellemezhetjük
egy $T\colon \R \to \R$ függvénnyel.
Így a $t\in\R$ időpontban a kemence hőmérsékletét
a $T(t)$ valós szám adja meg.

Egyszerre több változó modellezésére alkalmas a vektor függvény.
Például, ha a kemencében lévő nyomást ($P$) is számbavesszük, az
\[X:=\begin{bmatrix}T \\ P\end{bmatrix}\]
vektor függvény egy alkalmas matematikai model lehet a kemence állapotának leírására,
hiszen az $X(t)\in\R^2$ vektor megadja a kemence hőmérsékletét és nyomását tetszőleges $t\in\R$ időpontban.
\end{frame}



\begin{frame}[t]{Differenciálszámítás}
Alapvető fontosságú az alábbi kifejezések ismerete és azok gyakorlatban való alkalmazása!
\begin{table}
\centering
\begin{tabular}{ccc}
\toprule
$f(x)$ & $f'(x)$ \\
\midrule
$x^n$ & $n x^{n-1}$\\
$e^x$ & $e^x$\\
$\ln(x)$ & $\frac{1}{x}$\\
$\sin(x)$ & $\cos(x)$\\
$\cos(x)$ & $-\sin(x)$\\
\bottomrule
\end{tabular}
\end{table}

\[(cf)' = cf', \quad (fg)' = f'g + fg', \quad \left(\frac{f}{g}\right)' = \frac{f'g - fg'}{g^2}\]
\[(f\circ g)' = (f'\circ g) g', \quad \left(f^{-1}\right)' = \frac{1}{f'\circ f^{-1}}\]
\end{frame}

\begin{frame}[t]{Példa: logaritmus}
Számítsuk ki a logaritmus függvény deriváltját, mint az exponenciális függvény inverzének deriváltja.

Legyen $f(x):=e^x\ (x\in\R),$ aminek inverze $f^{-1}(x) = \ln(x)\ (x\in\R^+).$
Alkalmazva az $f$ függvényt annak inverzére
\[x = f(f^{-1}(x)) = e^{\ln(x)} \quad x\in\R^+,\]
majd deriválva mindkét oldalt kapjuk, hogy $1 = e^{\ln(x)} (\ln(x))',$ 
amiből a logaritmus deriváltja kifejezhető:
\[(\ln(x))' = \frac{1}{e^{\ln(x)}} = \frac{1}{x}  \quad x\in\R^+.\]
\begin{exampleblock}{Feladat}
Számoljuk ki az $\ln(g(x))$ deriváltját!
\end{exampleblock}
\end{frame}



\section{Lineáris Algebra}
\begin{frame}[t]{Lineáris kombináció}
A $v_1, v_2, v_3, \dots,v_N \in \R^n$ vektorok lineáris kombinációja:
\[c_1 v_1 + c_2 v_2 + c_3 v_3 + \dots + c_N v_N \in \R^n,\]
a $c_1, c_2, c_3,\dots, c_N\in\R$ együttatókkal.

\end{frame}

\section{Komplex Számok}
\begin{frame}[t]{Komplex számok}
\begin{columns}[t]
\begin{column}{0.48\linewidth}
{\bf Komplex számok alakjai}
\[z = x + iy = r\cos(\varphi) + ir\sin(\varphi)\]
\[x = r\cos(\varphi) \quad y = r\sin(\varphi)\]
\[r^2 = x^2 + y^2 \quad \tan(\varphi) = \frac{y}{x}\]
\begin{block}{Euler formula}
 \[e^{i\varphi} = \cos(\varphi) + i\sin(\varphi)\]
\end{block}
{\bf Komplex konjugált}
\[z^* = a - ib = r e^{-i\varphi}\]
\[z z^* = |z|^2 \quad \frac{z + z*}{2} = a \quad \frac{z - z*}{2} = b\]
\end{column}

\begin{column}{0.48\linewidth}
{\bf Műveletek}
\[z = a + ib = re^{i\varphi}, \quad w = c + id = se^{i\theta}\]
\[z \pm w = (a+c) \pm i (b+d)\] 
\[z w = (ac - bd) + i (ad + bc) = rse^{i(\varphi + \theta)}\]
\[\frac{1}{z} = \frac{z^*}{|z|^2} = \frac{a-ib}{a^2 + b^2} = \frac{1}{r} e^{-i\varphi}\]
\begin{block}{Egységgyökök}
A $z^n = 1$ egyenlet megoldásai:
\[z_k = \cos\left(2\pi\frac{k}{n}\right) + i\sin\left(2\pi\frac{k}{n}\right)\]
$k=0,1,2,\dots,n-1.$ 
\end{block}
\end{column}
\end{columns}    
\end{frame}

\begin{frame}
\begin{center}
\resizebox{!}{\textheight}{%
\begin{tikzpicture}[scale=4.0,cap=round,>=latex]
 % Unit circle
% Author: Supreme Aryal
% A unit circle with cosine and sine values for some
% common angles.
\documentclass[standalone]{article}
\usepackage{tikz}
%%%<
\usepackage{verbatim}
\usepackage[active,tightpage]{preview}
\PreviewEnvironment{tikzpicture}
\setlength\PreviewBorder{5pt}%
%%%>

\usepackage[sfdefault, lining]{FiraSans}
\renewcommand*\oldstylenums[1]{{\firaoldstyle #1}}
\usepackage{newtxsf}
\usepackage{microtype}

\begin{comment}
:Title: Unit circle

A unit circle with cosine and sine values for some common angles.
\end{comment}

\usepackage[top=1in,bottom=1in,right=1in,left=1in]{geometry}
\begin{document}
    \begin{tikzpicture}[scale=4.0,cap=round,>=latex]
        % draw the coordinates
        \draw[->] (-1.5cm,0cm) -- (1.5cm,0cm) node[right,fill=white] {$x$};
        \draw[->] (0cm,-1.5cm) -- (0cm,1.5cm) node[above,fill=white] {$y$};

        % draw the unit circle
        \draw[thick] (0cm,0cm) circle(1cm);

        \foreach \x in {0,30,...,360} {
                % lines from center to point
                \draw[gray] (0cm,0cm) -- (\x:1cm);
                % dots at each point
                \filldraw[black] (\x:1cm) circle(0.4pt);
                % draw each angle in degrees
                \draw (\x:0.6cm) node[fill=black!2] {$\x^\circ$};
        }

        % draw each angle in radians
        \foreach \x/\xtext in {
            30/\frac{\pi}{6},
            45/\frac{\pi}{4},
            60/\frac{\pi}{3},
            90/\frac{\pi}{2},
            120/\frac{2\pi}{3},
            135/\frac{3\pi}{4},
            150/\frac{5\pi}{6},
            180/\pi,
            210/\frac{7\pi}{6},
            225/\frac{5\pi}{4},
            240/\frac{4\pi}{3},
            270/\frac{3\pi}{2},
            300/\frac{5\pi}{3},
            315/\frac{7\pi}{4},
            330/\frac{11\pi}{6},
            360/2\pi}
                \draw (\x:0.85cm) node[fill=white] {$\xtext$};

        \foreach \x/\xtext/\y in {
            % the coordinates for the first quadrant
            30/\frac{\sqrt{3}}{2}/\frac{1}{2},
            45/\frac{\sqrt{2}}{2}/\frac{\sqrt{2}}{2},
            60/\frac{1}{2}/\frac{\sqrt{3}}{2},
            % the coordinates for the second quadrant
            150/-\frac{\sqrt{3}}{2}/\frac{1}{2},
            135/-\frac{\sqrt{2}}{2}/\frac{\sqrt{2}}{2},
            120/-\frac{1}{2}/\frac{\sqrt{3}}{2},
            % the coordinates for the third quadrant
            210/-\frac{\sqrt{3}}{2}/-\frac{1}{2},
            225/-\frac{\sqrt{2}}{2}/-\frac{\sqrt{2}}{2},
            240/-\frac{1}{2}/-\frac{\sqrt{3}}{2},
            % the coordinates for the fourth quadrant
            330/\frac{\sqrt{3}}{2}/-\frac{1}{2},
            315/\frac{\sqrt{2}}{2}/-\frac{\sqrt{2}}{2},
            300/\frac{1}{2}/-\frac{\sqrt{3}}{2}}
                \draw (\x:1.25cm) node[fill=white] {$\left(\xtext,\y\right)$};

        % draw the horizontal and vertical coordinates
        % the placement is better this way
        \draw (-1.25cm,0cm) node[above=1pt] {$(-1,0)$}
              (1.25cm,0cm)  node[above=1pt] {$(1,0)$}
              (0cm,-1.25cm) node[fill=white] {$(0,-1)$}
              (0cm,1.25cm)  node[fill=white] {$(0,1)$};
    \end{tikzpicture}
\end{document}

\end{tikzpicture}
}
\end{center}
\end{frame}

\begin{frame}[t]{}
\begin{exampleblock}{Feladat}
Határozzuk meg az alábbi komplex számok algebrai alakját:
\[ \frac{1}{i}, \quad \dfrac{1}{3-4i}, \quad e^{i2020\pi}, \quad ie^{i\frac{\pi}{2}}\]
\end{exampleblock}

\begin{exampleblock}{Feladat}
Oljduk meg az alábbi egyenleteket ($z\in\C$)
\[z^2 - 2z + 5 = 0, \quad z^3 = 1, \quad z^i = -1.\]
\end{exampleblock}

\begin{exampleblock}{Feladat}
Fejezzük ki a $\cos(3x)$-et $\cos(x)$ polinomjaként:
\[\cos(3x) = a_0 + a_1 \cos(x) + a_2 \cos^2(x) + a_3 \cos^3(x).\]
{\tiny Alkalmazzuk az Euler formulát az $e^{i3x} = (e^{ix})^3$ egyenletre, majd hasonlítsuk össze az egyenlet két oldalát.}
\end{exampleblock}
\end{frame}
\end{document}
