\documentclass[10pt]{beamer}
%\usepackage[utf8]{inputenc}

\usetheme[progressbar=frametitle]{metropolis}
\usepackage{appendixnumberbeamer}

\usepackage{booktabs}
\usepackage[scale=2]{ccicons}

\usepackage{amsmath, amssymb, amsfonts}

\usepackage{pgfplots}
\usepgfplotslibrary{dateplot}

\usepackage{xspace}
\newcommand{\themename}{\textbf{\textsc{metropolis}}\xspace}

\usepackage[sfdefault, lining]{FiraSans}
\renewcommand*\oldstylenums[1]{{\firaoldstyle #1}}
\usepackage{newtxsf}
\usepackage{microtype}

\newcommand{\R}[0]{\mathbb{R}}
\newcommand{\bp}[0]{\mathbf{p}}
\newcommand{\bv}[0]{\mathbf{v}}
\newcommand{\bF}[0]{\mathbf{F}}
\DeclareMathOperator{\sign}{sign}

\title{Differenciálegyenletek}
\subtitle{Dimenzióanalízis}
% \date{\today}
\date{}
\author{Csikja Rudolf}
\institute{Budapesti Műszaki és Gazdaságtudományi Egyetem\\Matemaikai Intézet, Analízis Tanszék}
% \titlegraphic{\hfill\includegraphics[height=1.5cm]{logo.pdf}}

\begin{document}

\maketitle

%\begin{frame}{Table of contents}
%  \setbeamertemplate{section in toc}[sections numbered]
%  \tableofcontents%[hideallsubsections]
%\end{frame}


\section{Idő transzformáció}
\begin{frame}[t]{Idő transzformáció}
\[x'(t) = f(x(t))\]
\[\tau = \varphi(t)\]

\[X := x\circ \varphi^{-1} \quad \to \quad X' = 
\frac{x'\circ \varphi^{-1}}{\varphi' \circ \varphi^{-1}} = 
\frac{f\circ x\circ \varphi^{-1}}{\varphi' \circ \varphi^{-1}} = 
\frac{f\circ X}{\varphi' \circ \varphi^{-1}}\]


\end{frame}

\end{document}
